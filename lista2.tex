% Options for packages loaded elsewhere
\PassOptionsToPackage{unicode}{hyperref}
\PassOptionsToPackage{hyphens}{url}
\PassOptionsToPackage{dvipsnames,svgnames,x11names}{xcolor}
%
\documentclass[
  letterpaper,
  DIV=11,
  numbers=noendperiod]{scrartcl}

\usepackage{amsmath,amssymb}
\usepackage{iftex}
\ifPDFTeX
  \usepackage[T1]{fontenc}
  \usepackage[utf8]{inputenc}
  \usepackage{textcomp} % provide euro and other symbols
\else % if luatex or xetex
  \usepackage{unicode-math}
  \defaultfontfeatures{Scale=MatchLowercase}
  \defaultfontfeatures[\rmfamily]{Ligatures=TeX,Scale=1}
\fi
\usepackage{lmodern}
\ifPDFTeX\else  
    % xetex/luatex font selection
\fi
% Use upquote if available, for straight quotes in verbatim environments
\IfFileExists{upquote.sty}{\usepackage{upquote}}{}
\IfFileExists{microtype.sty}{% use microtype if available
  \usepackage[]{microtype}
  \UseMicrotypeSet[protrusion]{basicmath} % disable protrusion for tt fonts
}{}
\makeatletter
\@ifundefined{KOMAClassName}{% if non-KOMA class
  \IfFileExists{parskip.sty}{%
    \usepackage{parskip}
  }{% else
    \setlength{\parindent}{0pt}
    \setlength{\parskip}{6pt plus 2pt minus 1pt}}
}{% if KOMA class
  \KOMAoptions{parskip=half}}
\makeatother
\usepackage{xcolor}
\setlength{\emergencystretch}{3em} % prevent overfull lines
\setcounter{secnumdepth}{-\maxdimen} % remove section numbering
% Make \paragraph and \subparagraph free-standing
\ifx\paragraph\undefined\else
  \let\oldparagraph\paragraph
  \renewcommand{\paragraph}[1]{\oldparagraph{#1}\mbox{}}
\fi
\ifx\subparagraph\undefined\else
  \let\oldsubparagraph\subparagraph
  \renewcommand{\subparagraph}[1]{\oldsubparagraph{#1}\mbox{}}
\fi


\providecommand{\tightlist}{%
  \setlength{\itemsep}{0pt}\setlength{\parskip}{0pt}}\usepackage{longtable,booktabs,array}
\usepackage{calc} % for calculating minipage widths
% Correct order of tables after \paragraph or \subparagraph
\usepackage{etoolbox}
\makeatletter
\patchcmd\longtable{\par}{\if@noskipsec\mbox{}\fi\par}{}{}
\makeatother
% Allow footnotes in longtable head/foot
\IfFileExists{footnotehyper.sty}{\usepackage{footnotehyper}}{\usepackage{footnote}}
\makesavenoteenv{longtable}
\usepackage{graphicx}
\makeatletter
\def\maxwidth{\ifdim\Gin@nat@width>\linewidth\linewidth\else\Gin@nat@width\fi}
\def\maxheight{\ifdim\Gin@nat@height>\textheight\textheight\else\Gin@nat@height\fi}
\makeatother
% Scale images if necessary, so that they will not overflow the page
% margins by default, and it is still possible to overwrite the defaults
% using explicit options in \includegraphics[width, height, ...]{}
\setkeys{Gin}{width=\maxwidth,height=\maxheight,keepaspectratio}
% Set default figure placement to htbp
\makeatletter
\def\fps@figure{htbp}
\makeatother

\usepackage{booktabs}
\usepackage{longtable}
\usepackage{array}
\usepackage{multirow}
\usepackage{wrapfig}
\usepackage{float}
\usepackage{colortbl}
\usepackage{pdflscape}
\usepackage{tabu}
\usepackage{threeparttable}
\usepackage{threeparttablex}
\usepackage[normalem]{ulem}
\usepackage{makecell}
\usepackage{xcolor}
\usepackage[auth-lg]{authblk}
\KOMAoption{captions}{tableheading}
\makeatletter
\makeatother
\makeatletter
\makeatother
\makeatletter
\@ifpackageloaded{caption}{}{\usepackage{caption}}
\AtBeginDocument{%
\ifdefined\contentsname
  \renewcommand*\contentsname{Table of contents}
\else
  \newcommand\contentsname{Table of contents}
\fi
\ifdefined\listfigurename
  \renewcommand*\listfigurename{List of Figures}
\else
  \newcommand\listfigurename{List of Figures}
\fi
\ifdefined\listtablename
  \renewcommand*\listtablename{List of Tables}
\else
  \newcommand\listtablename{List of Tables}
\fi
\ifdefined\figurename
  \renewcommand*\figurename{Figure}
\else
  \newcommand\figurename{Figure}
\fi
\ifdefined\tablename
  \renewcommand*\tablename{Table}
\else
  \newcommand\tablename{Table}
\fi
}
\@ifpackageloaded{float}{}{\usepackage{float}}
\floatstyle{ruled}
\@ifundefined{c@chapter}{\newfloat{codelisting}{h}{lop}}{\newfloat{codelisting}{h}{lop}[chapter]}
\floatname{codelisting}{Listing}
\newcommand*\listoflistings{\listof{codelisting}{List of Listings}}
\makeatother
\makeatletter
\@ifpackageloaded{caption}{}{\usepackage{caption}}
\@ifpackageloaded{subcaption}{}{\usepackage{subcaption}}
\makeatother
\makeatletter
\@ifpackageloaded{tcolorbox}{}{\usepackage[skins,breakable]{tcolorbox}}
\makeatother
\makeatletter
\@ifundefined{shadecolor}{\definecolor{shadecolor}{rgb}{.97, .97, .97}}
\makeatother
\makeatletter
\makeatother
\makeatletter
\makeatother
\ifLuaTeX
  \usepackage{selnolig}  % disable illegal ligatures
\fi
\IfFileExists{bookmark.sty}{\usepackage{bookmark}}{\usepackage{hyperref}}
\IfFileExists{xurl.sty}{\usepackage{xurl}}{} % add URL line breaks if available
\urlstyle{same} % disable monospaced font for URLs
\hypersetup{
  pdftitle={Lista 2},
  pdfauthor={César Augusto Galvão - 19/0011572; Laiza Mendes - 20/0067028},
  colorlinks=true,
  linkcolor={blue},
  filecolor={Maroon},
  citecolor={Blue},
  urlcolor={Blue},
  pdfcreator={LaTeX via pandoc}}

\title{Lista 2}
\usepackage{etoolbox}
\makeatletter
\providecommand{\subtitle}[1]{% add subtitle to \maketitle
  \apptocmd{\@title}{\par {\large #1 \par}}{}{}
}
\makeatother
\subtitle{Modelos Lineares Generalizados - 2/2023}
\author{César Augusto Galvão - 19/0011572 \and Laiza Mendes -
20/0067028}
\date{}

\begin{document}
\maketitle
\ifdefined\Shaded\renewenvironment{Shaded}{\begin{tcolorbox}[breakable, interior hidden, boxrule=0pt, borderline west={3pt}{0pt}{shadecolor}, enhanced, frame hidden, sharp corners]}{\end{tcolorbox}}\fi

\renewcommand*\contentsname{Table of contents}
{
\hypersetup{linkcolor=}
\setcounter{tocdepth}{2}
\tableofcontents
}
\newpage{}

\hypertarget{questuxe3o-1}{%
\section{Questão 1}\label{questuxe3o-1}}

Considere os dados sobre a qualidade do vinho tinto, apresentados no
ficheiro \texttt{Q01-data.txt}. Ajuste o modelo de regressão linear
múltipla, e faça uma análise completa desses dados. Que conclusões você
tira dessa análise? (use 5\% de significância durantes as análises).

\hypertarget{a-proponha-algum-muxe9todo-para-resolver-o-problema-da-multicolinearidade-no-conjunto-de-dados}{%
\subsection{a) Proponha algum método para resolver o problema da
multicolinearidade no conjunto de
dados}\label{a-proponha-algum-muxe9todo-para-resolver-o-problema-da-multicolinearidade-no-conjunto-de-dados}}

\hypertarget{b-usando-algum-muxe9todo-de-seleuxe7uxe3o-de-variuxe1veis-obtenha-o-modelo-final-para-o-conjunto-de-dados}{%
\subsection{b) Usando algum método de seleção de variáveis, obtenha o
modelo final para o conjunto de
dados}\label{b-usando-algum-muxe9todo-de-seleuxe7uxe3o-de-variuxe1veis-obtenha-o-modelo-final-para-o-conjunto-de-dados}}

\hypertarget{c-apresente-a-tabela-de-anuxe1lise-de-variuxe2ncia-para-testar-a-significuxe2ncia-global-dos-coeficientes-do-modelo-final.-apresente-as-hipuxf3teses-de-teste-e-conclua.}{%
\subsection{c) Apresente a tabela de Análise de Variância para testar a
significância global dos coeficientes do modelo final. Apresente as
hipóteses de teste e
conclua.}\label{c-apresente-a-tabela-de-anuxe1lise-de-variuxe2ncia-para-testar-a-significuxe2ncia-global-dos-coeficientes-do-modelo-final.-apresente-as-hipuxf3teses-de-teste-e-conclua.}}

\hypertarget{d-com-base-no-modelo-obtido-no-item-anterior-fauxe7a-uma-anuxe1lise-de-resuxedduos-e-conclua.}{%
\subsection{d) Com base no modelo obtido no item anterior, faça uma
análise de resíduos e
conclua.}\label{d-com-base-no-modelo-obtido-no-item-anterior-fauxe7a-uma-anuxe1lise-de-resuxedduos-e-conclua.}}

\hypertarget{questuxe3o-2}{%
\section{Questão 2}\label{questuxe3o-2}}

Uma equipe de pesquisadores de saúde mental deseja comparar três métodos
de tratamento da depressão grave (A, B e C=referência). Eles também
gostariam de estudara relação entre idade e eficácia do tratamento, bem
como a interação (se houver) entre idade e tratamento. Cada elemento da
amostra aleatória simples de 36 pacientes, foi selecionado
aleatoriamente para receber o tratamento A, B ou C. Os dados obtidos
podem ser encontrados no ficheiro \texttt{Q02-data.txt}. A variável
dependente \(y\) é a eficácia do tratamento; as variáveis independentes
são: a idade do paciente no aniversário mais próximo e o tipo de
tratamento administrado (use 1\% de significância durantes as análises).

Uma amostra dos dados é exibida na tabela a seguir:

\begin{longtable*}{ccc}
\toprule
eficacia & idade & tratamento\\
\midrule
\endfirsthead
\multicolumn{3}{@{}l}{\textit{(continued)}}\\
\toprule
eficacia & idade & tratamento\\
\midrule
\endhead

\endfoot
\bottomrule
\endlastfoot
\cellcolor{gray!15}{56} & \cellcolor{gray!15}{21} & \cellcolor{gray!15}{A}\\
41 & 23 & B\\
\cellcolor{gray!15}{40} & \cellcolor{gray!15}{30} & \cellcolor{gray!15}{B}\\
28 & 19 & C\\
\cellcolor{gray!15}{55} & \cellcolor{gray!15}{28} & \cellcolor{gray!15}{A}\\
25 & 23 & C\\*
\end{longtable*}

\hypertarget{a-ajuste-um-modelo-de-regressuxe3o-linear-e-interprete-os-resultados-obtidos}{%
\subsection{a) Ajuste um modelo de regressão linear e interprete os
resultados
obtidos}\label{a-ajuste-um-modelo-de-regressuxe3o-linear-e-interprete-os-resultados-obtidos}}

Temos um potencial modelo de regressão linear que pode ou não conter
interações entre as variáveis, o qual pode ser expresso em sua forma
saturada, em que \(X_1\) é a variável idade e \(X_2\) a variável
tratamento

\begin{align}
  y_i = \beta_0 + \beta_1 \, x_{1i} + \beta_2 \, x_{2i} + \beta_3 \, x_{1i}\, x_{2i} + \varepsilon_i, \quad i = 1, 2, \dots, n
\end{align}

ou, de forma análoga, desmembrando \(X_2\) em variáveis \emph{dummy}
\(X_A\) e \(X_B\), indicadores da presença do tratamento \(A\) e \(B\),
ambas assumindo valor \(0\) quando se trata do tratamento \(C\)

\begin{align}
  y_i = \beta_0 + \beta_1 \, x_{1i} + \beta_2 \, x_{Ai} + \beta_3 \, x_{Bi} + \beta_4 \, x_{1i} \, x_{Ai} + \beta_5 \, x_{1i} \, x_{Bi} + \varepsilon_i. \label{modelo_dummy}
\end{align}

Se simplesmente ajustamos um modelo de regressão linear -- sem os termos
de interação -- utilizando (\ref{modelo_dummy}) como referência na
função \texttt{lm()}, obtemos os seguintes resultados:

\begin{longtable*}{lcccc}
\toprule
Coeficiente & Estimativa & EP & Estatística t & p-valor\\
\midrule
\endfirsthead
\multicolumn{5}{@{}l}{\textit{(continued)}}\\
\toprule
Coeficiente & Estimativa & EP & Estatística t & p-valor\\
\midrule
\endhead

\endfoot
\bottomrule
\endlastfoot
\cellcolor{gray!15}{(Intercept)} & \cellcolor{gray!15}{22.291} & \cellcolor{gray!15}{3.505} & \cellcolor{gray!15}{6.359} & \cellcolor{gray!15}{0.000}\\
idade & 0.664 & 0.070 & 9.522 & 0.000\\
\cellcolor{gray!15}{A} & \cellcolor{gray!15}{10.253} & \cellcolor{gray!15}{2.465} & \cellcolor{gray!15}{4.159} & \cellcolor{gray!15}{0.000}\\
B & 0.445 & 2.464 & 0.181 & 0.858\\*
\end{longtable*}

Ou seja, se considerarmos independentemente idade, tratamento A e
tratamento B, podemos considerar que:

\begin{itemize}
\tightlist
\item
  Há uma linha de base na eficácia de aproximadamente 22.3, i.e.~sob o
  tratamento C;
\item
  A eficácia base para o tratamento A é de 32.3;
\item
  A eficácia base para o tratamento B é de 22.75 -- mas poderíamos
  desconsiderar este coeficiente, se nos guiarmos pelo p-valor;
\item
  Cada ano a mais de vida incrementa a eficácia em 0.644.
\end{itemize}

É possível considerar que um tamanho de amostra pequeno tenha grande
influência sobre a significância de \(H_0: \beta_3 = 0\) do modelo. No
entanto, trata-se de um fenômeno para o qual o tratamento pode estar
estreitamente associado à idade, caso em que teríamos que considerar o
modelo (\ref{modelo_dummy}) por completo.

\hypertarget{b-obtenha-a-tabela-anova-para-o-modelo-obtido-no-item-a-e-interprete-os-resultado}{%
\subsection{b) Obtenha a tabela ANOVA para o modelo obtido no item (a) e
interprete os
resultado}\label{b-obtenha-a-tabela-anova-para-o-modelo-obtido-no-item-a-e-interprete-os-resultado}}

\hypertarget{c-considere-a-possibilidade-de-incluir-a-interauxe7uxe3o-entre-as-varuxe1veis-independentes}{%
\subsection{c) Considere a possibilidade de incluir a interação entre as
varáveis
independentes}\label{c-considere-a-possibilidade-de-incluir-a-interauxe7uxe3o-entre-as-varuxe1veis-independentes}}

Supõe-se que \(\varepsilon_i \sim N(0, \sigma^2)\).

\hypertarget{i-lista-de-todos-os-submodelos-possuxedveis}{%
\subsubsection{i) Lista de todos os submodelos
possíveis}\label{i-lista-de-todos-os-submodelos-possuxedveis}}

A partir do modelo (\ref{modelo_dummy}), construimos todos os possíveis
submodelos. Considerando que temos três covariáveis e dois termos de
interação, temos \(\sum\limits_{n = 1}^5\binom{6}{n} = 62\) modelos

\begin{enumerate}
  \item $y_i = \beta_0 + \varepsilon_i$
  \item $y_i = \beta_1 \, x_{1i} + \varepsilon_i$
  \item $y_i = \beta_2 \, x_{Ai} + \varepsilon_i$
  \item $y_i = \beta_3 \, x_{Bi} + \varepsilon_i$
  \item $y_i = \beta_4 \, x_{1i} \, x_{Ai} + \varepsilon_i$
  \item $y_i = \beta_5 \, x_{1i} \, x_{Bi} + \varepsilon_i$
  
 
  \item $y_i = \beta_0 + \beta_1 \, x_{1i} + \varepsilon_i$
  \item $y_i = \beta_0 + \beta_2 \, x_{Ai} + \varepsilon_i$
  \item $y_i = \beta_0 + \beta_3 \, x_{Bi} + \varepsilon_i$
  \item $y_i = \beta_0 + \beta_4 \, x_{1i} \, x_{Ai} + \varepsilon_i$
  \item $y_i = \beta_0 + \beta_5 \, x_{1i} \, x_{Bi} + \varepsilon_i$
  \item $y_i = \beta_1 \, x_{1i} + \beta_2 \, x_{Ai} + \varepsilon_i$
  \item $y_i = \beta_1 \, x_{1i} + \beta_3 \, x_{Bi} + \varepsilon_i$
  \item $y_i = \beta_1 \, x_{1i} + \beta_4 \, x_{1i} \, x_{Ai} + \varepsilon_i$
  \item $y_i =  \beta_1 \, x_{1i} + \beta_5 \, x_{1i} \, x_{Bi} + \varepsilon_i$
  \item $y_i = \beta_2 \, x_{Ai} + \beta_3 \, x_{Bi}+ \varepsilon_i$
  \item $y_i = \beta_2 \, x_{Ai} + \beta_4 \, x_{1i} \, x_{Ai} + \varepsilon_i$
  \item $y_i = \beta_2 \, x_{Ai} + \beta_5 \, x_{1i} \, x_{Bi} + \varepsilon_i$
  \item $y_i = \beta_3 \, x_{Bi} + \beta_4 \, x_{1i} \, x_{Ai} + \varepsilon_i$
  \item $y_i = \beta_3 \, x_{Bi} \beta_5 \, x_{1i} \, x_{Bi} + \varepsilon_i$
  \item $y_i = \beta_4 \, x_{1i} \, x_{Ai} + \beta_5 \, x_{1i} \, x_{Bi} + \varepsilon_i$
  
  

  \item $y_i = \beta_0 + \beta_1 \, x_{1i} + \beta_2 \, x_{Ai} + \varepsilon_i$
  \item $y_i = \beta_0 + \beta_1 \, x_{1i} + \beta_3 \, x_{Bi} + \varepsilon_i$
  \item $y_i = \beta_0 + \beta_1 \, x_{1i} + \beta_4 \, x_{1i} \, x_{Ai} + \varepsilon_i$
  \item $y_i = \beta_0 + \beta_1 \, x_{1i} + \beta_5 \, x_{1i} \, x_{Bi} + \varepsilon_i$
  \item $y_i = \beta_0 + \beta_2 \, x_{Ai} + \beta_3 \, x_{Bi} + \varepsilon_i$
  \item $y_i = \beta_0 + \beta_2 \, x_{Ai} + \beta_4 \, x_{1i} \, x_{Ai} +  \varepsilon_i$
  \item $y_i = \beta_0 + \beta_2 \, x_{Ai} + \beta_5 \, x_{1i} \, x_{Bi} + \varepsilon_i$
  \item $y_i = \beta_0 + \beta_3 \, x_{Bi} + \beta_4 \, x_{1i} \, x_{Ai} +  \varepsilon_i$
  \item $y_i = \beta_0 + \beta_3 \, x_{Bi} + \beta_5 \, x_{1i} \, x_{Bi} + \varepsilon_i$
  \item $y_i = \beta_0 + \beta_4 \, x_{1i} \, x_{Ai} + \beta_5 \, x_{1i} \, x_{Bi} + \varepsilon_i$
  \item $y_i = \beta_1 \, x_{1i} + \beta_2 \, x_{Ai} + \beta_3 \, x_{Bi} + \varepsilon_i$
  \item $y_i = \beta_1 \, x_{1i} + \beta_2 \, x_{Ai} + \beta_4 \, x_{1i} \, x_{Ai} + \varepsilon_i$
  \item $y_i = \beta_1 \, x_{1i} + \beta_2 \, x_{Ai} + \beta_5 \, x_{1i} \, x_{Bi} + \varepsilon_i$
  \item $y_i = \beta_1 \, x_{1i} + \beta_3 \, x_{Bi} + \beta_4 \, x_{1i} \, x_{Ai} + \varepsilon_i$
  \item $y_i = \beta_1 \, x_{1i} + \beta_3 \, x_{Bi} + \beta_5 \, x_{1i} \, x_{Bi} + \varepsilon_i$
  \item $y_i = \beta_1 \, x_{1i} + \beta_4 \, x_{1i} \, x_{Ai} + \beta_5 \, x_{1i} \, x_{Bi} + \varepsilon_i$
  \item $y_i = \beta_2 \, x_{Ai} + \beta_3 \, x_{Bi} + \beta_4 \, x_{1i} \, x_{Ai} + \varepsilon_i$
  \item $y_i = \beta_2 \, x_{Ai} + \beta_3 \, x_{Bi} + \beta_5 \, x_{1i} \, x_{Bi} + \varepsilon_i$
  \item $y_i = \beta_2 \, x_{Ai} + \beta_4 \, x_{1i} \, x_{Ai} + \beta_5 \, x_{1i} \, x_{Bi} + \varepsilon_i$
  \item $y_i = \beta_3 \, x_{Bi} + \beta_4 \, x_{1i} \, x_{Ai} + \beta_5 \, x_{1i} \, x_{Bi} + \varepsilon_i$
  

 
  \item $y_i = \beta_0 + \beta_1 \, x_{1i} + \beta_2 \, x_{Ai} + \beta_3 \, x_{Bi} + \varepsilon_i$
  \item $y_i = \beta_0 + \beta_1 \, x_{1i} + \beta_2 \, x_{Ai} + \beta_4 \, x_{1i} \, x_{Ai} + \varepsilon_i$
  \item $y_i = \beta_0 + \beta_1 \, x_{1i} + \beta_2 + \beta_5 \, x_{1i} \, x_{Bi} + \varepsilon_i$
  \item $y_i = \beta_0 + \beta_1 \, x_{1i} + \beta_3 \, x_{Bi} + \beta_4 \, x_{1i} \, x_{Ai} + \varepsilon_i$
  \item $y_i = \beta_0 + \beta_1 \, x_{1i} + \beta_3 \, x_{Bi} + \beta_5 \, x_{1i} \, x_{Bi} + \varepsilon_i$
  \item $y_i = \beta_0 + \beta_1 \, x_{1i} + \beta_4 \, x_{1i} \, x_{Ai} + \beta_5 \, x_{1i} \, x_{Bi} + \varepsilon_i$
  
  \item $y_i = \beta_0 + \beta_2 \, x_{Ai} + \beta_3 \, x_{Bi} + \beta_4 \, x_{1i} \, x_{Ai} + \varepsilon_i$
  \item $y_i = \beta_0 + \beta_2 \, x_{Ai} + \beta_3 \, x_{Bi} + \beta_5 \, x_{1i} \, x_{Bi} + \varepsilon_i$
  \item $y_i = \beta_0 + \beta_2 \, x_{Ai} + \beta_4 \, x_{1i} \, x_{Ai} + \beta_5 \, x_{1i} \, x_{Bi} + \varepsilon_i$
  \item $y_i = \beta_0 + \beta_3 \, x_{Bi} + \beta_4 \, x_{1i} \, x_{Ai} + \beta_5 \, x_{1i} \, x_{Bi} + \varepsilon_i$
  
  \item $y_i = \beta_1 \, x_{1i} + \beta_2 \, x_{Ai} + \beta_3 \, x_{Bi} + \beta_4 \, x_{1i} \, x_{Ai} + \varepsilon_i$
  \item $y_i = \beta_1 \, x_{1i} + \beta_2 \, x_{Ai} + \beta_3 \, x_{Bi} + \beta_5 \, x_{1i} \, x_{Bi} + \varepsilon_i$
  \item $y_i = \beta_1 \, x_{1i} + \beta_2 \, x_{Ai} + \beta_4 \, x_{1i} \, x_{Ai} + \beta_5 \, x_{1i} \, x_{Bi} + \varepsilon_i$
  \item $y_i = \beta_1 \, x_{1i} + \beta_3 \, x_{Bi} + \beta_4 \, x_{1i} \, x_{Ai} + \beta_5 \, x_{1i} \, x_{Bi} + \varepsilon_i$
  
  \item $y_i = \beta_2 \, x_{Ai} + \beta_3 \, x_{Bi} + \beta_4 \, x_{1i} \, x_{Ai} + \beta_5 \, x_{1i} \, x_{Bi} + \varepsilon_i$
  
  \item $y_i = \beta_1 \, x_{1i} + \beta_2 \, x_{Ai} + \beta_3 \, x_{Bi} + \beta_4 \, x_{1i} \, x_{Ai} + \beta_5 \, x_{1i} \, x_{Bi} + \varepsilon_i$
  
  \item $y_i = \beta_0 + \beta_2 \, x_{Ai} + \beta_3 \, x_{Bi} + \beta_4 \, x_{1i} \, x_{Ai} + \beta_5 \, x_{1i} \, x_{Bi} + \varepsilon_i$
  \item $y_i = \beta_0 + \beta_1 \, x_{1i} + \beta_3 \, x_{Bi} + \beta_4 \, x_{1i} \, x_{Ai} + \beta_5 \, x_{1i} \, x_{Bi} + \varepsilon_i$
  \item $y_i = \beta_0 + \beta_1 \, x_{1i} + \beta_2 \, x_{Ai} + \beta_4 \, x_{1i} \, x_{Ai} + \beta_5 \, x_{1i} \, x_{Bi} + \varepsilon_i$
  \item $y_i = \beta_0 + \beta_1 \, x_{1i} + \beta_2 \, x_{Ai} + \beta_3 \, x_{Bi} + \beta_5 \, x_{1i} \, x_{Bi} + \varepsilon_i$
  \item $y_i = \beta_0 + \beta_1 \, x_{1i} + \beta_2 \, x_{Ai} + \beta_3 \, x_{Bi} + \beta_4 \, x_{1i} \, x_{Ai} + \varepsilon_i$
\end{enumerate}

\hypertarget{ii-interpretauxe7uxe3o-de-coeficientes-de-regressuxe3o-de-fatores-de-interauxe7uxe3o}{%
\subsubsection{ii) Interpretação de coeficientes de regressão de fatores
de
interação}\label{ii-interpretauxe7uxe3o-de-coeficientes-de-regressuxe3o-de-fatores-de-interauxe7uxe3o}}

\hypertarget{iii-tabela-anova}{%
\subsubsection{iii) tabela ANOVA}\label{iii-tabela-anova}}

\hypertarget{iv-anuxe1lise-completa-dos-resuxedduos-do-modelo}{%
\subsubsection{iv) Análise completa dos resíduos do
modelo}\label{iv-anuxe1lise-completa-dos-resuxedduos-do-modelo}}

\hypertarget{apuxeandice}{%
\section{Apêndice}\label{apuxeandice}}

Todo o projeto de composição deste documento pode ser encontrado aqui:
\url{https://github.com/cesar-galvao/mlg}



\end{document}
